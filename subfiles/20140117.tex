\documentclass[../main.tex]{subfiles}
\begin{document}

\subsection{Pairwise Swap}
\begin{question}
Consider p processes (p is even), each process has 1 data element. The even processes s swap their element with their neighbour s+1 (= pair-wise swap).
\begin{enumerate}
	\item Give a BSP algorithm.
	\item What is the $h$-relation?
	\item Suppose (non-BSP) 2-sided send and receive. When send transmits, hold execution until remote process receives. Then execute the communication and both processes continue. This is blocking pair-wise communication (as featured in MPI). Give the algorithm.
	\item Executing a send-receive communication has start-up time $s$, and data is transfered with bandwidth $b$. Is $s < l$? Will the cost be higher than the BSP algorithm from 1.1?
\end{enumerate}
\end{question}
\begin{solution} The solution to the subquestions are given below.
\begin{enumerate}
	\item The BSP program for one-to-all communication is given below.
\begin{lstlisting}
void pairwiseSwap(){
	bsp_begin(P);
	bsp_push_reg(element);
	int s = bsp_pid();
	if(s % 2 == 0){
		bsp_put(s+1, data, element, 0, sizeof(data));
	}else if{
		bsp_put(s-1, data, element, 0, sizeof(data));
	}
	bsp_sync();
	bsp_end();
}
\end{lstlisting}
	\item The $h$-relation is given by $h = max\{h_s,h_r\}$. In the previous case, a processor sends one element and receives one. Therefore, $h = 1$.
	\item The non-BSP program for one-to-all communication is given below. We assume primitives \texttt{send(message, dest)} and \texttt{receive(message, sender)}.
\begin{lstlisting}
void pairwiseSwap(){
	init_comm(s,P);

	if(s % 2 == 0){
		send(s+1, data);
		receive(data, sendproc);
		if(sendproc == s + 1)
			ERROR
	}else if{
		send(s-1, data;
		receive(data, sendproc);
		if(sendproc == s - 1)
			ERROR
	}
}
\end{lstlisting}
\end{enumerate}
\end{solution}

\subsection{SpMV}
\begin{question}
$A$ is a sparse $m$ by $n$ matrix, $x$ and $y$ are vectors.
\begin{enumerate}
	\item What is the sequential cost of $y=Ax$?
	\item Assume $A$ is distributed row-wise in a 1D fashion. Which phase, if any, of the classic SpMV phases (fan-out, SpMV, fan-in) will disappear? What will be the new total cost? What will be the parallel overhead?
	\item $H=(V,N)$ is a hypergraph of A. What hypergraph model would you use to distribute A using the previous 1D distribution type? Give definitions of $V$ and $N$ in that model.
	\item Assume SpMV algorithm 4.5 as in the book. Use your definition of $H$ from the previous question to write a cost function that measures the number of gets and puts.
\end{enumerate}
\end{question}
\begin{solution}
\end{solution}

\subsection{Odd-Even Transposition}
\begin{question}
Array $x$ has $n$ elements with $n$ a multiple of $p$. In odd-even transposition sort, the fundamental operation is compare-exchange (when each process has 1 element, $n = p$) or compare-split (each process has $n/p$ elements). Give for both cases:
\begin{enumerate}
	\item A BSP algorithm.
	\item Parallel execution time.
	\item Parallel efficiency.
	\item For n/p elements: if n increases (with a fixed p), what will happen to the parallel efficiency?
\end{enumerate}



\end{question}
\begin{solution}
\end{solution}

\subsection{Shared Memory}
\begin{question}
We have p threads running the same SPMD program on a shared memory computer. Will the following algorithm work? Describe any possible problems. Supply a working parallel algorithm.
\begin{lstlisting}[caption={SPMD program on a shared memory computer},label=lst:spmd]
double a;
double x[1000];
double y[1000];

void spmd(){
	for (i=0; i<1000; i+=p)
		a = a + x[i]y[i];
}
\end{lstlisting}

\end{question}
\begin{solution}
\end{solution}

\end{document}